\documentclass[12pt,a4paper]{ctexart}
\usepackage[utf8]{inputenc}
\usepackage{amsmath}
\usepackage{amsfonts}
\usepackage{amssymb}
\usepackage{graphicx}
\usepackage{wallpaper}
\author{阴月有晴}
\title{概率论笔记}
\date{国庆}
\begin{document}
\maketitle
\begin{center}
\CenterWallPaper{1.2}{back.jpg}
手机:18576427500\\
邮箱1:chenay@sustc.edu.cn\\
邮箱2:achen@liv.ac.uk
\end{center}
\tableofcontents

\section{Probability}
\subsection{simple concepts}
$ \Omega  $代表sample space,$ \omega  $代表基本事件,$ f $代表$ 2^{\Omega} $,其中有$ 2^{\left| \Omega \right|} $个元素

大写字母ABC代表事件,大写字母XYZ代表随机变量

$ \phi $:例如抛四次硬币,不可能出现反反反。

\paragraph{Operation of Eventl}
\begin{itemize}
	\item Union 并 $ \cup $
	\item Intersection 交 $ \cap $
	\item complement 德摩根律  $ A \backslash B $ 
	\item contain $ A \subset B $
	\item disjoint A和B不可能同时发生
	\item \textbf{概率测度}:$ P:f\rightarrow \left[ 0,1\right]  $,\textit{不可能事件概率为零,概率为零的事件未必为不可能事件。必然事件亦同}
\end{itemize}

概率空间:$ \left( \Omega,f,p\right)  $

\subsection{Computing Probabitity}
条件概率,定义,可推广至多个。

\textbf{Polya's urn schenme} 一个例子

\textbf{School boy}又一个例子



Law of Total Probability: Let$ B_{1},B_{2},B_{i},.. $为$ \Omega $的一个划分

由全概率公式可进一步推出贝叶斯公式。

独立性的含义。必然事件和零概率事件与任何事件都独立。分析定义和直观定义。并且也可以进一步推广(由两个事件推广至n个事件)。

\section{Random Variables}
随机变量的概念:$ \Omega \rightarrow \mathbb{R}  $,将事件的结果与数相关,赋予事件以意义。在数这方面可以进行推广。

Bernouli Random variable,一种很常见的模型。

Binomial Random variable,是Bernouli由一次推广至n次的结果,也可以看作是其的应用。

Poisson Random variable Poisson($ \lambda $)

Geometric Random variable 超几何分布

反过来,可以用随机变量的结果来代表事件。

随机变量由其核心性质可分为离散型核心变量(无穷或又穷,可列)和连续性随机变量。

\textbf{p.m.f}:相当于根据随机变量的结果在概率空间上的划分。

\textbf{c.d.f}(cumulative dtribution function):一个非降函数,离散型随机变量会有跳跃点,右连续和左极限。
大概是反映了一种顺序吧,类似一种格的结构。

一些离散型随机变量的分布,如泊松分布,伯努利分布等。


\section{Continuous Random Varaibles}
连续性随机变量:不可列的无穷变量。

类似p.m.f可以在连续性随机变量定义\textbf{p.d.f}(概率密度)。(\textbf{更好的想法是从c.d.f上定义出来,即要求$ F(x) $可微,若不可微则为第三类随机变量}),


而c.d.f则可以直接套用在连续性随机变量上,名称依然为c.d.f没有改变

p.d.f在某些点可以取不同的值。这只能从c.d.f这方面来理解。

$ p( a < x\leq b) =F\left( b\right) -F\left( a\right)  $,该公式对于所有类型的随机变量均有效。且应该注意左开右闭,对于连续性随机变量则无影响。且$ a > b $时右边为零而非反过来

The Exponential Distribution

The Gammma Distribution :可以看作Exponential Distribution 的一种更近一步的推广。

\textbf{Normal Random Variabls}:$ X \sim N\left( 0,1\right)  $,
$f(x)= \dfrac {1}{\sqrt {2\pi }}e^{-\dfrac {x^{2}}{2}} $,其c.d.f又记为\[ \Phi \left(x \right) =\int_{-\infty}^{x} f \left(x \right) dx\]
由于$ f(x) $关于y轴的对称性,$ \Phi \left(x\right) $有性质:
\[ \Phi \left(x\right)=1-\Phi \left(-x\right) \]
其中,可以引出\textbf{$ 3 \sigma $原则}

\textbf{General Normal Random Variables:}
\[ X \sim N \left(\mu,\sigma ^2\right) \]
\[ f\left(x\right) =\dfrac{1}{\sqrt{2 \pi} \sigma} e^{-\dfrac{\left(x-\mu\right)^2}{2 \sigma ^2}}\]
令\[ Y=a X+b \] 则
\[ Y \sim N \left(a\mu+b,a^2 \sigma ^2\right) \]
令\[ Y=\dfrac{X-\mu}{\sigma} \]
then \[ Y \sim N \left(0,1\right) \]

\textbf{function of a random variable}:$ R \rightarrow R $,例如:
\[ z=x^k,then \, Z \left(\omega\right)= X^k \, is \, a \, r.v\]
\textbf{question } if $ X \sim N \left(0,1\right) $,then what is the distribution of $ X^2 $?\\
Solution : Let $ Y=X^2 $ and assume the c.d.f of Y is $ F_Y \left(y\right) $.
\textbf{answer }:
\[ f_Y \left(y\right)=\begin{cases}
\dfrac{e^{-\dfrac{y}{2}}}{\sqrt{2 \pi y}}  \, , \, y>0\\\\

0,\,y \leq 0
\end{cases} \]

\textbf{a useful general result}:Suppose that $ g \left(x\right) $ is a strictly monotone differentiable function of x.Then the random variable Y denoted by Y=$ g \left(x\right) $ has a p.d.f. given by
\[  f_{Y} \left(y\right) =\begin{cases}
f_X \left[g^-1 \left(y\right)\right] \cdot   \left[\dfrac{d}{dy} g^-1 \left(y\right)\right],\, if \, y=g\left(x\right)\\
\\
0,\,if \,there \,is \, no \,y=g \left(x\right)
\end{cases}\]

\textbf{the cauthy distribution}:$ \mu $ is a real number,if its p.d.f. $ f_Y \left(y\right) $is given by:
\[ f_Y \left(y\right)=\dfrac{1}{\pi} \cdot \dfrac{1}{\left[1+\left(y-\mu\right)^2\right]},\left(-\infty<y<\infty\right)\] 
when $ \mu =0 $ is the standard cauthy distribution.

\section{Expected Values}
\subsection{The Expected Value of a  Discrete Random Variable}
\paragraph{Definition of expexted}$ E \left(X\right) $,表示随机变量的对于概率的加权平均。
\[ E \left(X\right)=\sum_{i}^{} p_{i}\cdot X_{i}\]
其中若上式为无穷级数,则要求其绝对收敛,否则无意义(黎曼重排定理)

\paragraph{Binomial Distribution}if $ X \sim B \left(n,p\right) $,then\[ E\left(X\right)=np \]
\paragraph{Poisson Distribution }if $ X \sim Possion \left(\lambda\right) $,then\[ E \left(X\right)=\lambda \]

\paragraph{The Geometric Distribution} $ E \left(X\right)=\dfrac{1}{p} $,  $ p $代表一次成功的概率。

\paragraph{notes} $ E \left(X\right)  $ does no exist \[ p.m.f \, p_n=\dfrac{1}{n\left(n+1\right)}   \]

\subsection{The Expected Value of a  Continuous Random Variable}
\paragraph{Definition of expexted}类比离散型随机变量,
\[ E \left(X\right)=\int_{-\infty}^{\infty} x f\left(x\right) \, dx\]

\paragraph{Standard Normal Distribution}
由对称性和\textbf{绝对收敛}可知,Standard Normal Distribution 的$
 E\left(X\right) =0$
 
\paragraph{Gamma Distribution}
利用$ \Gamma  $函数的性质可以演算出:
\[ E \left(X\right)=\dfrac{\alpha}{\lambda} \]

\paragraph{The Cauthy Distribution}
The cauthy distribution is 
\[ f \left(x\right)=\dfrac{1}{\pi} \cdot \dfrac{1}{1+x^2}\]
but  \[ \int_{-\infty}^{\infty} \left|x\right| f \left(x\right) \, dx\]does not convert,so its $  E \left(X\right)$ does not exist.

\paragraph{Remark}

\subsection{Expectation of Function of Random Variables}

\paragraph{Problem}
Suppose X is a r.v. ,the $ Y=g \left(X\right) $,how to find the $ E \left(Y\right) $?

\paragraph{Notes}
\begin{itemize}
	\item 为什么$ E \left(Y\right)\neq g \left( E \left(X\right)\right) $
	\item 其中的意义如何理解?
	\item n阶矩的意义?
\end{itemize}
\paragraph{Examples}
Random Variable Random Variable X:{-1,0,1}\\
with pmf:P(X=-1)=0.2 , P(X=0)=0.5 ,P(X=1)=0.3\\
then\[ E \left(X^2\right)=0.5 \]

\subsection{二阶矩}

\paragraph{Notes}
\begin{itemize}
	\item $ E \left( X^2\right)-\left(E \left(X\right)\right)^2 $是方差(如何证明?)
	\item 矩是研究随机变量的一种工具,类似于物理上的矩。
\end{itemize}
\paragraph{Question}
\begin{itemize}
	\item 如何理解$ E \left(X ^2\right) $的意义?为什么被赋予二阶矩的意义?
	\item 是否有更快的计算方法?
	\item 对于连续性变量的方差如何定义?
	\item 如何理解$ E \left(X-Y\right)= E \left(X\right) -E \left(Y\right) $的意义?
\end{itemize}

\section{Joint Distribution}
\subsection{离散型随机变量}
\paragraph{Joint p.m.f}
假设X和Y是定义在同一个样本空间上的离散型随机变量,则
\[ p \left(x_i,y_j\right) = P \left( X=x_i,Y=y_i\right) \]
(其中涉及交运算),称为X和Y的 the joint probability mass function ,简写为Joint p.m.f

\paragraph{Marginal p.m.f}
$ p_X \left(\cdot\right) $称为X和Y的 the joint probability mass function 的X的边缘分布

\paragraph{M p.m.f 和 Joint p.m.f的关系}
显然Joint p.m.f可以决定M p.m.f,但是如何决定?而已知M p.m.f,在什么情况下可以决定Joint p.m.f?\\
第一问利用全概率公式即可,类似一种降维的思想\\
而第二问则需要引入\textbf{独立性随机变量}的概念。因为需要知道X和Y的关系。

\paragraph{独立性随机变量}
若对于所有的$ x_i $和$ y_i $有:
\[ P \left(x_i,y_j\right) =P \left(x_i\right) \cdot P \left(y_j\right)\]
,联想独立事件,可以看作一个独立事件族。

\paragraph{Joint Cumulative Probability Distribution Function}
简写为:joint c.d.f,记作$ F_{\left(X,Y\right)} \left(x,y\right)$,
,有二维递增性和概率连续性:\[ \lim\limits_{x \rightarrow-\infty} F \left(x,y\right)=0 ,\forall y \in R\]

\paragraph{Marginal Cumulative Probability Distribution Function}

\paragraph{两者c.d.f的关系}
类似c.d.f,但是计算方法有所改变。

\subsection{连续性随机变量}
类似单变量随机变量,先建立Joint c.d.f,它有二维递增性和积分为1,以及区间
性。
再由偏导可以进一步定义 Joint p.d.f (Joint Probability Density Function)以及 M p.d.f

\textbf{Remark}:独立性和相关系数的关系,相关系数为0未必独立,而独立相关系数一定为0.

\subsection{多随机变量的函数的期望}
多随机变量的函数,例如$ Z =X+Y $,是一个复合函数,也是一个 r.v.,仍然以事件为因变量。以正向的思维来看,这个是简单的
\subsection{期望的性质}
\begin{enumerate}
\item \textbf{$ E \left(C\right)=C $,C为一常数},你的初始理解有错误
\item $ E \left(a X\right)=a \cdot E \left(X\right) $,a为常数,X为随机变量
\item $ E \left(X \cdot Y\right) \neq E \left(X\right) \cdot E \left(Y\right) $,并不显然
\item $ E \left(X+Y\right)=E \left(X\right)+E \left(Y\right) $,要假设其存在p.d.f,然后由积分可证
\item $ E \left(aX+bY\right)=a \cdot E \left(X\right)+b \cdot E 
\left(Y\right) $,是2和4的推论,代表了期望的线性性质。
\item $ E \left( \sum_{n=1}^{\infty} Y_n\right)= \sum_{n=1}^{\infty}  E \left(Y_n\right)$,其中要保证无穷项符号要一致\\
控制收敛定理:若$ \sum_{n=1}^{\infty} E \left(\left|Y_n \right|\right) $有限,则前式也成立
\item 若X和Y两者独立(上面的定理不需要这个条件),那么$ E \left(XY\right)=E \left(X\right) \cdot E \left(Y\right) $,证明过程
中只需要利用$ f \left(x,y\right)=f_X \left(x\right) \cdot f_Y \left(y\right) $套积分公式即可\\
进一步推广,\[ E \left(g \left(X\right) \cdot h \left(Y\right)\right)
=E \left(g \left(X\right)\right) \cdot E \left(h \left(Y\right)\right) \]
\end{enumerate}

\paragraph{Remark }数学归纳法,只对有限数有用,不能过渡到无穷。(超限归纳法)

\subsection{方差}
因为$ E \left(X\right) $无法反映随机变量的所有性质,无法反映其“spread”

$ E \left|X-E \left(X\right)\right| $可以用来测定“spread",但不好计算

为了在数学上更好的处理,故用$ E \left(\left(X-E \left(X\right)\right)^2\right) $
来定义方差

\[ Var \left(X\right)=E \left(X^2\right) -\left(E \left(
X\right)\right)^2 \]

这个公式的证明体现了这个体系的优越性.均方差为方差的开方。

\paragraph{方差的性质}
\begin{enumerate}
	\item 方差大于零
	\item $ Var \left(C\right) =0$,C为常数
	\item $ Var \left(a \cdot X\right) =a^2 \cdot Var \left(X\right)$,不是线性性质,但利用了期望的线性性质
	\item 如果X和Y互相独立,有:
	\[ Var \left(X+Y\right) =Var \left(X\right)+Var \left(Y\right)\]
	反映了方差的线性性质,这里利用了期望的第三条性质,虽然形式上
	跟第二条性质相似
	\item a,b为常数,则
	\[ Var \left(a+bX\right)=b^2 \cdot Var \left(X\right) \]
\end{enumerate}

\subsection{协方差}
Covariance:
\[ Cov \left(X,Y\right) =E \left(\left(X-E \left(X\right)\right)
\left(Y-\left(E \left(Y\right)\right)\right) \right)\]
它的性质蛮多的。不过均是可以由条件期望的三条性质推导出来的。
它可以取任意实数.,代表一种相关性

可以引出一个Varaince Formula,蛮有用的

\subsection{相关系数}
由于Covariance可能太大,故引进Correlation Cofficient: $ \rho 
\left(X,Y\right) $

其中两个方差严格大于零.

可以构造证明\[ -1 \leqq \rho \leqq 1 \]

如果$ \rho =1 $,则两者呈现正线性关系。

\textbf{不过对于这个线性关系怎样理解?}

另外,可以与线性代数来使n个随机变量的Covariance 形式更漂亮

\subsection{Moment Generating Function}
矩发生函数,MGF
\[ M_X \left(t\right)=E \left(e ^{tX}\right) \]
是关于t的实函数。

显然,
\[ M_X \left(t\right)=E \left(e ^{tX}\right)=
\int_{-\infty}^{\infty} e ^{tx} f \left(x\right) dx\]
由于绝对收敛性,所以可能对某些t不成立

矩发生函数的性质:
\begin{enumerate}
	\item $ M_X\left(0\right) =1$
	\item If $ Y=aX+b $,$ M_Y \left(t\right)=e^{bt} M_X \left(at\right)$,利用期望的线性性质即可。
	\item If Z=X+Y ,and X,Y are independent then $ 
	M_Z \left(t\right)=M_X \left(t\right) \cdot M_Y \left(t\right) $,可拓展至 n 个独立随机变量
	\item mgf与其随机变量一一对应。拉氏变换的唯一性
\end{enumerate}

\textbf{Remark} 你对几何分布的认识有问题。

矩发生函数的应用:
\begin{enumerate}
	\item 用来求随机变量的矩:
	\[ M'_X \left(t\right)=\dfrac{d}{dt} E \left(e^{tX}\right)=E \left(\dfrac{d}{dt} e^{tX}\right)=
	E \left(e^{tX}X\right)\]
	则:\[ E \left(X^n\right)=M_X^{(n)} \left(0\right)\]
	称为原点矩,另外还有均值矩$ E\left[\left(X-E \left(X \right) \right)^n\right]  $,\textbf{也是可以求出来的}
	\item 用来决定分布
	利用性质三和性质四可以求组合独立随机变量的分布
\end{enumerate}

\textbf{Remark} $ Y \sim \Gamma \left(\dfrac{1}{2},\dfrac{n}{2}\right) $,其中$ Y \sim \chi ^2 \left(n\right) $

\section{条件概率}
\subsection{引入概念}
首先从离散型随机变量引入。并且与Joint p.m.f和Marginal p.m.f
\paragraph{Conditional p.m.f}
是x的函数,y在其中起参数的作用,\[ p_{X|Y} \left(x|y\right)\]
\paragraph{Conditional c.d.f}
由Conditional p.m.f 求和而得,仍然是x的函数
\[ F_{X|Y} \left(x|y\right)\]
\paragraph{条件期望}
\[ E \left(X|Y=y\right)=\sum_{x} x \cdot p_{X|Y} 
\left(x|y\right)
\]
其中要求$ p_Y{y} >0$,这也是难以向连续性随机变量拓展的困难之处。

\paragraph{an example}
如果X和Y是独立的泊松分布,求\[ E \left(X|X+Y=n\right) \]
answer is $ n \dfrac{\lambda _{1}}{\lambda _{2}+\lambda_{1}} $

\paragraph{连续性随机变量的条件期望的引入}
取极限就是
\[ f_{X|Y} \left(x|y\right)=\dfrac{f \left(x,y\right)}{f_Y \left(y\right)}其中f_Y \left(y\right)>0\]
\[ E \left(X|Y=y\right)=\int_{-\infty}^{\infty} x \cdot f_{X|Y} \left(x|y\right) /, dx \]
是y的函数

\end{document}
